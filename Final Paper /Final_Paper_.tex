% Template for PLoS
% Version 3.5 March 2018
%
% % % % % % % % % % % % % % % % % % % % % %
%
% -- IMPORTANT NOTE
%
% This template contains comments intended
% to minimize problems and delays during our production
% process. Please follow the template instructions
% whenever possible.
%
% % % % % % % % % % % % % % % % % % % % % % %
%
% Once your paper is accepted for publication,
% PLEASE REMOVE ALL TRACKED CHANGES in this file
% and leave only the final text of your manuscript.
% PLOS recommends the use of latexdiff to track changes during review, as this will help to maintain a clean tex file.
% Visit https://www.ctan.org/pkg/latexdiff?lang=en for info or contact us at latex@plos.org.
%
%
% There are no restrictions on package use within the LaTeX files except that
% no packages listed in the template may be deleted.
%
% Please do not include colors or graphics in the text.
%
% The manuscript LaTeX source should be contained within a single file (do not use \input, \externaldocument, or similar commands).
%
% % % % % % % % % % % % % % % % % % % % % % %
%
% -- FIGURES AND TABLES
%
% Please include tables/figure captions directly after the paragraph where they are first cited in the text.
%
% DO NOT INCLUDE GRAPHICS IN YOUR MANUSCRIPT
% - Figures should be uploaded separately from your manuscript file.
% - Figures generated using LaTeX should be extracted and removed from the PDF before submission.
% - Figures containing multiple panels/subfigures must be combined into one image file before submission.
% For figure citations, please use "Fig" instead of "Figure".
% See http://journals.plos.org/plosone/s/figures for PLOS figure guidelines.
%
% Tables should be cell-based and may not contain:
% - spacing/line breaks within cells to alter layout or alignment
% - do not nest tabular environments (no tabular environments within tabular environments)
% - no graphics or colored text (cell background color/shading OK)
% See http://journals.plos.org/plosone/s/tables for table guidelines.
%
% For tables that exceed the width of the text column, use the adjustwidth environment as illustrated in the example table in text below.
%
% % % % % % % % % % % % % % % % % % % % % % % %
%
% -- EQUATIONS, MATH SYMBOLS, SUBSCRIPTS, AND SUPERSCRIPTS
%
% IMPORTANT
% Below are a few tips to help format your equations and other special characters according to our specifications. For more tips to help reduce the possibility of formatting errors during conversion, please see our LaTeX guidelines at http://journals.plos.org/plosone/s/latex
%
% For inline equations, please be sure to include all portions of an equation in the math environment.
%
% Do not include text that is not math in the math environment.
%
% Please add line breaks to long display equations when possible in order to fit size of the column.
%
% For inline equations, please do not include punctuation (commas, etc) within the math environment unless this is part of the equation.
%
% When adding superscript or subscripts outside of brackets/braces, please group using {}.
%
% Do not use \cal for caligraphic font.  Instead, use \mathcal{}
%
% % % % % % % % % % % % % % % % % % % % % % % %
%
% Please contact latex@plos.org with any questions.
%
% % % % % % % % % % % % % % % % % % % % % % % %

\documentclass[10pt,letterpaper]{article}
\usepackage[top=0.85in,left=2.75in,footskip=0.75in]{geometry}

% amsmath and amssymb packages, useful for mathematical formulas and symbols
\usepackage{amsmath,amssymb}

% Use adjustwidth environment to exceed column width (see example table in text)
\usepackage{changepage}

% Use Unicode characters when possible
\usepackage[utf8x]{inputenc}

% textcomp package and marvosym package for additional characters
\usepackage{textcomp,marvosym}

% cite package, to clean up citations in the main text. Do not remove.
% \usepackage{cite}

% Use nameref to cite supporting information files (see Supporting Information section for more info)
\usepackage{nameref,hyperref}

% line numbers
\usepackage[right]{lineno}

% ligatures disabled
\usepackage{microtype}
\DisableLigatures[f]{encoding = *, family = * }

% color can be used to apply background shading to table cells only
\usepackage[table]{xcolor}

% array package and thick rules for tables
\usepackage{array}

% create "+" rule type for thick vertical lines
\newcolumntype{+}{!{\vrule width 2pt}}

% create \thickcline for thick horizontal lines of variable length
\newlength\savedwidth
\newcommand\thickcline[1]{%
  \noalign{\global\savedwidth\arrayrulewidth\global\arrayrulewidth 2pt}%
  \cline{#1}%
  \noalign{\vskip\arrayrulewidth}%
  \noalign{\global\arrayrulewidth\savedwidth}%
}

% \thickhline command for thick horizontal lines that span the table
\newcommand\thickhline{\noalign{\global\savedwidth\arrayrulewidth\global\arrayrulewidth 2pt}%
\hline
\noalign{\global\arrayrulewidth\savedwidth}}


% Remove comment for double spacing
%\usepackage{setspace}
%\doublespacing

% Text layout
\raggedright
\setlength{\parindent}{0.5cm}
\textwidth 5.25in
\textheight 8.75in

% Bold the 'Figure #' in the caption and separate it from the title/caption with a period
% Captions will be left justified
\usepackage[aboveskip=1pt,labelfont=bf,labelsep=period,justification=raggedright,singlelinecheck=off]{caption}
\renewcommand{\figurename}{Fig}

% Use the PLoS provided BiBTeX style
% \bibliographystyle{plos2015}

% Remove brackets from numbering in List of References
\makeatletter
\renewcommand{\@biblabel}[1]{\quad#1.}
\makeatother



% Header and Footer with logo
\usepackage{lastpage,fancyhdr,graphicx}
\usepackage{epstopdf}
%\pagestyle{myheadings}
\pagestyle{fancy}
\fancyhf{}
%\setlength{\headheight}{27.023pt}
%\lhead{\includegraphics[width=2.0in]{PLOS-submission.eps}}
\rfoot{\thepage/\pageref{LastPage}}
\renewcommand{\headrulewidth}{0pt}
\renewcommand{\footrule}{\hrule height 2pt \vspace{2mm}}
\fancyheadoffset[L]{2.25in}
\fancyfootoffset[L]{2.25in}
\lfoot{\today}

%% Include all macros below

\newcommand{\lorem}{{\bf LOREM}}
\newcommand{\ipsum}{{\bf IPSUM}}





\usepackage{forarray}
\usepackage{xstring}
\newcommand{\getIndex}[2]{
  \ForEach{,}{\IfEq{#1}{\thislevelitem}{\number\thislevelcount\ExitForEach}{}}{#2}
}

\setcounter{secnumdepth}{0}

\newcommand{\getAff}[1]{
  \getIndex{#1}{}
}

\providecommand{\tightlist}{%
  \setlength{\itemsep}{0pt}\setlength{\parskip}{0pt}}

\begin{document}
\vspace*{0.2in}

% Title must be 250 characters or less.
\begin{flushleft}
{\Large
\textbf\newline{Webscrabing Doctors' Information from the United States Immigration
website as a first step to track down doctors and facilities that have
been harassing immigrants} % Please use "sentence case" for title and headings (capitalize only the first word in a title (or heading), the first word in a subtitle (or subheading), and any proper nouns).
}
\newline
% Insert author names, affiliations and corresponding author email (do not include titles, positions, or degrees).
\\
Rutendo Madziwo\textsuperscript{\getAff{Smith College}},
Maggie Szlosek\textsuperscript{\getAff{Smith College}},
Rachel LaFlamme\textsuperscript{\getAff{Smith College}}\\
\bigskip
\bigskip
\end{flushleft}
% Please keep the abstract below 300 words
\section*{Abstract}
The title of the project and a one-paragraph abstract of the entire
project with recommended length of no more than 150 words.

% Please keep the Author Summary between 150 and 200 words
% Use first person. PLOS ONE authors please skip this step.
% Author Summary not valid for PLOS ONE submissions.

\linenumbers

% Use "Eq" instead of "Equation" for equation citations.
\emph{Text based on plos sample manuscript, see
\url{http://journals.plos.org/ploscompbiol/s/latex}}

\section{Introduction}\label{introduction}

\section{Maggie's}\label{maggies}

The research question(s) Background/significance of the research

\section{Method}\label{method}

We collected the doctors' data using the package rvest for web-scraping,
RSelenium for web navigation and an external platform Docker for virtual
interaction with the web browser.

Docker is a platform to develop, deploy, and run applications inside
containers{[}{\textbf{???}}{]}. We were able to virtually interact with
the USCIS website by connecting to a port in Docker and opening Chrome
and this enabled us to control and see what was happening on the website
at a given time. Initially, Docker was installed before installing and
loading the needed R Packages. The command
\texttt{docker\ run\ -d\ -p\ 4445:4444\ selenium/standalone-chrome} was
then run in the R Terminal after we had installed our packages. This
command sets up the virtual Chrome container to enable interaction with
the Chrome web browser. In order to check if Docker is running, one can
type in \texttt{docker\ ps}. We eventually open the browser using
RSelenium commands before scraping our data.

RSelenium is a package in R which helps one connect to a Selenium
server. This server in turn connects to the Chrome web browser and hence
allowed us to automate our webscraping experience. RSelenium is
responsible not only for opening and closing the browser, but it allowed
us to virtually navigate the web page and automatically control the
scraping. This was especially useful as our website had no endpoint urls
and hence could not rely on more traditional web scraping methods. In
addition, it made the process of scraping the data faster as one can
simply allow the code to run and scrape multiple pages without needing
to manually click the specific website.

While RSelenium was responsible for most of the web manouvering, the
package we used for scraping the data from each of the pages was
\texttt{rvest}. This package makes harvesting data from a website easy
as it can find specific html nodes, and their children. It also allows
one to use both XPaths and CSS selectors so though we eventually stuck
to using basic elements, we were not limited to one option. As a side
note, we chose to use CSS selectors for web navigation with the
RSelenium package.

We created a function to scrape this data and took advantage of the
purrr package in R to map all our scraped elements together. To clean up
our data, \texttt{dplyr} and \texttt{tidyverse} were used for
text-processing the such that zipcodes were in a separate column from
the rest of the address in the resulting doctors' dataset. At the
moment, this function is running in a for loop but will be converted to
a while loop in order to allow for different state scenarios.

The final code written to collect the doctors' information allows a user
to input one zipcode at a time in order to scrape data. Once that
zipcode is entered, the doctors and facilities on that web page are
harvested using \texttt{rvest} before moving on to the next page.
Clicking to the next page has been automated using \texttt{RSelenium}
and a for loop was implemented in our code such that for a certain
number of times, the website's \texttt{Next} button is clicked, moves on
to the next page, scrapes that page and so on. The website itself has
been written in such a way that an actual user can keep clicking to find
the nearest doctors within a 500 miles radius. As such, we have also
manually entered different zipcodes in different parts of the USA so as
to capture all the doctors in the country and create different datasets.

\section{Maggie's}\label{maggies-1}

A discussion of the research, the limitations of the current research,
reasonableness of any assumptions made, possibilities of future
work/studies that should be conducted, etc.

Here are two sample references: {[}1,2{]}.

\section*{References}\label{references}
\addcontentsline{toc}{section}{References}

\hypertarget{refs}{}
\hypertarget{ref-Feynman1963118}{}
1. Feynman R, Vernon Jr. F. The theory of a general quantum system
interacting with a linear dissipative system. Annals of Physics.
1963;24: 118--173.
doi:\href{https://doi.org/10.1016/0003-4916(63)90068-X}{10.1016/0003-4916(63)90068-X}

\hypertarget{ref-Dirac1953888}{}
2. Dirac P. The lorentz transformation and absolute time. Physica.
1953;19: 888--896.
doi:\href{https://doi.org/10.1016/S0031-8914(53)80099-6}{10.1016/S0031-8914(53)80099-6}

\nolinenumbers


\end{document}

